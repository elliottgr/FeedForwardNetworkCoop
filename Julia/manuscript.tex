\documentclass[11pt]{article}

%%%%%%%%%%%%%%%%%%
%% PACKAGE ZONE %%
%%%%%%%%%%%%%%%%%%



\begin{document}

%%%%%%%%%%%%%%%%%%
%% INTRODUCTION %%
%%%%%%%%%%%%%%%%%%

% Game theory, quant and pop gen

Though the role of natural selection in shaping social behavior has been puzzled over since Darwin \cite{darwin_origin_1882}, it was not until Hamilton's work in the 1960s that a comprehensive theory was offered \cite{hamilton_evolution_1963, hamilton_genetical_1964}. This provided a starting point for later theoretical approaches, but these later analyses often ignored genetic constraints and instead relied on the "phenotypic gambit" \cite{grafen_chapter_nodate}, which assumes that the long-term evolution of phenotypic optima is rarely constrained by genetic forces such as linkage disequilibrium or population structure \cite{eshel_changing_1996}.
The most noteworthy of these approaches is evolutionary game theory, which rose to prominance following Smith and Price's treatment of inter-organismal conflict \cite{smith_logic_1973}, particularly the notion of the evolutionarily stable strategy (ESS) \cite{smith_evolution_1982, dugatkin_game_1998}. Later work would begin to explicitly incorporate genetic considerations into models of social evolution by extending existing models of sexual selection developed by quantitative geneticists \cite{lande_sexual_1980, lande_models_1981, kirkpatrick_sexual_1982, wolf_interacting_1999}. These attempts to extend quantitative genetics towards social evolution led to the development of "social selection" models \cite{wolf_interacting_1999}, where the fitness of a focal individual is a consequence of traits expressed by other members of the population. \par
Numerous attempts to synthesize evolutionary game theory and population or quantitative genetics have been proposed \cite{eshel_changing_1996, hammerstein_darwinian_1996, weissing_genetic_1996, aoki_quantitative_1983, aoki_quantitative_1984, gomulkiewicz_game_2000, lion_theoretical_2018, taylor_selection_1996, queller_quantitative_1992, queller_expanded_2011}, leading to a conception of the two approaches as differing in emphasis while still ultimately describing the same process \cite{mcglothlin_synthesis_2022, van_cleve_social_2015-1}. Under this framework, we conceive of genetic models as capturing the "short-term" of evolution \cite{lande_sexual_1980, eshel_changing_1996, moore_interacting_1997}, where evolution selects between a finite set of alleles and is potentially dominated by interacting genetic forces such as linkage, epistasis, or plasticity. Conversely, we can conceive of ESS models as the "long-term", with evolution primarily being thought of in terms of adaptation towards some phenotypic optimum, and largely unconstrained by genetic interactions \cite{mcnamara_game_2020, smith_evolution_1982}.

% network stuff, mechanisms of interaction

%%%%%%%%%%%%%%%%%%
%%    METHODS   %%
%%%%%%%%%%%%%%%%%%

%%% how do the networks work

% layer activation

% 

%%% how does the game work

% rounds

% payoff, benefit, and cost

%

%%% how does the population structure work


%%%%%%%%%%%%%%%%%%
%%    RESULTS   %%
%%%%%%%%%%%%%%%%%%

%%%%%%%%%%%%%%%%%%
%%  DISCUSSION  %%
%%%%%%%%%%%%%%%%%%

%%%%%%%%%%%%%%%%%%
%%  REFERENCES  %%
%%%%%%%%%%%%%%%%%%
\bibliographystyle{plain}
\bibliography{ffn_cites}

\end{document}